\documentclass[line, resMargin, a4paper]{res}
\usepackage[utf8]{inputenc}
\textheight=780pt
\usepackage{times}
\usepackage{xcolor}
\usepackage{enumitem}
\setlist{itemsep=0pt}
\setlist{leftmargin=10pt}
\usepackage[hidelinks]{hyperref}
\hypersetup{
    colorlinks,
    linkcolor={red!50!black},
    citecolor={blue!50!black},
    urlcolor={blue!80!black}
}

\begin{document}

\name{Lucas Kent}
\address{
    \href{https://github.com/rukai}{github.com/rukai}\\
    %50 Candowie Crescent\\ Baulkham Hills 2153\\ 
    Ph: +61 449 123 093\\
    Email: rubickent@gmail.com
}

\begin{resume}
    \section{Education}
    \begin{itemize}
        \item Bachelor of IT major in Software Technology - Macquarie University (2019)
    \end{itemize}

    \section{Skills}
    \begin{itemize}
        \item Languages: Rust, Python, C\#, C, C++, Java, Lua, SQL, HTML, CSS, Javascript
        \item Tools: Git, Vim, VS Code, Visual Studio, JetBrains IDEs
        \item Services: AWS, Github Actions
        %\item OS: Linux, Windows
    \end{itemize}

    \section{Proffessional Work}

    \textbf{\emph {Full-time Software Developer \href{https://www.instaclustr.com}{Instaclustr (acquired under NetApp) }}} (current employment, 2021-)
    \begin{itemize}
        \item I am one of two employees in the larger R\&D team working on the \href{https://github.com/shotover/shotover-proxy}{Shotover proxy} (rust)
        \item Shotover is an open source L7 database proxy
            \subitem Supports redis, cassandra and kafka
            \subitem Use cases include load balancing, data encryption and caching
            \subitem Custom query/response transformations can be written in rust
        \item \href{https://github.com/shotover/shotover-proxy/pulls?q=author%3Arukai}{My work on shotover} includes feature design, code/test implementation and PR reviews.
        \item I built \href{https://github.com/shotover/windsock}{windsock}, a DB benchmark framework to support evaluation of shotover performance changes.
        \item To support my work on shotover I frequently make contributions to open source dependencies when I find bugs or missing features e.g. \href{https://github.com/krojew/cdrs-tokio/pulls?q=author%3Arukai}{cdrs-tokio}, \href{https://github.com/tychedelia/kafka-protocol-rs/pulls?q=is%3Apr+is%3Aclosed+author%3Arukai}{kafka-protocol}
    \end{itemize}

    \textbf{\emph {Full-time Software Developer \href{http://www.wisetechglobal.com}{WiseTech Global}}} (2016-2021)
    \begin{itemize}
        \item Worked in the embedded team working on our truck telematics devices which report truck data such as GPS, fuel, on board mass and tyre pressure. (python, C)
        \item Worked on CargoWise One our main logistics management software. (C\#, TSQL, WinForms)
        \item Worked on the messaging system between CargoWise One and external services such as the customs system for various countries. (C\#, TSQL, BizTalk, XSLT)
        \item Worked on various web-based internal tools. (C\#, TSQL, ASP.NET, Javascript)
        \item Mentored multiple newstarters on working within my team. (6 months each)
    \end{itemize}

    \section{Hobby Projects}

    \textbf{\emph {Contributor to the Rust ecosystem}} (2017-2024)
    \begin{itemize}
        \item I have made \href{https://github.com/rust-lang/rust/pulls?q=author%3Arukai}{some contributions} to rustc itself
            \subitem Most of my contributions are to rust's diagnostics, my favorite part of the language
        \item I have made a lot of contributions to existing libraries in the Rust ecosystem including:
        \subitem \href{https://github.com/gfx-rs/wgpu/pulls?utf8=%E2%9C%93&q=author%3Arukai}{wgpu} - Implementation of WebGPU used by Firefox
        \subitem \href{https://github.com/rust-windowing/winit/pulls?utf8=%E2%9C%93&q=author%3Arukai+}{Winit} - Cross platform Rust windowing library
        \subitem \href{https://github.com/vulkano-rs/vulkano/pulls?utf8=%E2%9C%93&q=author%3Arukai}{Vulkano} - High level Rust vulkan bindings
        \item I have also made some libraries of my own:
        \subitem \href{https://github.com/rukai/winit_input_helper}{Winit Input Helper} - Processes winit events, allowing inputs to be queried at any time
        \subitem \href{https://github.com/rukai/ggbasm}{GGBASM} - Assembler for Gameboy accessed via a Rust library API
        \subitem \href{https://github.com/rukai/treeflection}{Treeflection} - Pseudo-reflection functionality for Rust
        \subitem \href{https://github.com/rukai/cargo-run-wasm}{cargo run-wasm} - Trivially run wasm applications and examples in the browser
    \end{itemize}

    \textbf{\emph {\href{https://github.com/rukai/DPedal}{DPedal} - USB directional pedal}} (2023)
    \begin{itemize}
        \item 3D printable mechanical parts designed in \href{https://cad.onshape.com/documents/b3650977a607511c32026f52/w/79027c5ddd8ad99ee7db1e2a/e/7192077cb58abe7f31bd20c3?renderMode=0&uiState=63ad8d5084623c01cce27891}{Onshape}.
        \item PCB designed in kicad with reference to an open source keyboard.
        \item Firmware written in rust, makes HID keyboard and mouse events in response to input.
        \item Custom flashing application writes the firmware and configuration to the device.
    \end{itemize}

    \newpage

    \textbf{\emph{Super Smash Bros related projects - Tools for smash players and my own engine/game}} (2016-2022)
    \begin{itemize}
        \item \href{https://github.com/rukai/brawllib_rs}{brawllib\_rs} - A Rust library for parsing and processing \href{https://en.wikipedia.org/wiki/Super_Smash_Bros._Brawl}{Brawl} character files.
        \item \href{https://rukaidata.com/PM3.6/Marth/subactions/Attack11.html}{rukaidata.com} - A framedata website for Brawl/Project M. Generator written in rust and frontend written in rust compiled to wasm, uses brawllib\_rs. Hosted on AWS using EC2, S3, Route53, and cloudfront.
        \item \href{https://github.com/rukai/PF_Sandbox}{Platform Fighter Sandbox} - An engine for smash-like games, it features a character editor tightly integrated with gameplay. Written in Rust. Can import characters from Brawl via brawllib\_rs.
        \item \href{https://canoncollision.com}{Canon Collision} - A fork of Platform Fighter Sandbox to allow me to focus on building an actual game without worrying about making a generic engine. I have notably implemented a GLTF 3D renderer since the fork.
    \end{itemize}

    \textbf{\emph {Contributor to \href{https://github.com/dolphin-emu/dolphin/pulls?q=author\%3Arukai}{Dolphin}}} (2015-2018)
    \begin{itemize}
        \item Dolphin is the most complete Gamecube/Wii emulator
        \item I contributed to the complete UI rewrite from wxWidgets to Qt.
    \end{itemize}

    \textbf{\emph {Built a slow and incomplete \href{https://github.com/rukai/GameToy}{Gameboy emulator} that boots several games}} (2016)

\end{resume}
\end{document}
